%Context / topic / problem overview
%Why is it important / motivations
%Applications
%Hypothesis
%What was done
%Main results

\abstract{The estimation of the optimal sensor constellation for maximizing the observable surface area of a given set of target objects is a challenging and combinatorial explosive problem that has a wide range of applications for perception tasks that may require gathering sensor information from multiple views due to environment occlusions. To tackle this problem, it was developed a system for accurate simulation of depth cameras, that modeled the main hardware characteristics, such as image resolution, field of view, range of measurements and acquisition rate. Later on, several populations of these sensors were deployed within 4 different testing environments targeting active perception and bin picking applications with increasing level of occlusions and geometry complexity. These sensor populations were either uniformly or randomly inserted on a set of regions of interest in which useful sensor data could be retrieved and in which the real sensors could be installed or moved by a robotic arm. The proposed approach managed to quickly estimate useful sensor constellations for maximizing the observable surface area of a set of target objects, making it suitable to be used for deciding the type and spatial disposition of sensors and also guide movable 3D cameras for avoiding dynamic environment occlusions.}
