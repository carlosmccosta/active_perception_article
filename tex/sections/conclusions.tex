\section{\uppercase{Conclusions}}\label{sec:conclusions}

\noindent The proposed approach managed to estimate suitable sensor constellations for maximizing the observable surface of a given set of target objects (starter motors) that were deployed on a trolley in several testing environments with increasing perception complexity and varying degree of occlusions. Each constellation had several types of depth cameras, that were modeling the main characteristics of eight commercially available sensors, such as depth image resolution, field of view, range of valid measurements and data acquisition rate. For making this combinatorial explosive problem computational tractable, each population of possible view points for each sensor type was either randomly or uniformly deployed over regions in which the real sensors could be placed and provide useful perception data. With a reasonable sensor count the system managed to compute the best sensor disposition in a few seconds, which makes it suitable for active perception tasks, such as active object tracking and bin picking.

Future work would include the testing of the proposed approach in conjunction with a object recognition system in order to reliably perform object tracking when an operator is manipulating a target object (by moving the sensor within the environment using a robotic arm). Further research for fast object tracking recovery could include the modeling of the interaction between the target objects and a simulated hand, in order to keep an approximate object pose even if it is almost completely occluded.
