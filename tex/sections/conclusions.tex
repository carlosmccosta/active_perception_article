\section{Conclusions}\label{sec:conclusions}

The proposed sensor placement system was able to generate sensor constellations for maximizing the observable surface area percentage of a given set of target objects (starter motors) that were deployed on a trolley in several testing environments with increasing perception complexity and varying degree of occlusions. Each constellation had several types of depth cameras, that were modeling the main characteristics of eight 3D sensors, such as depth image resolution, field of view, range of valid measurements and data acquisition rate. For making this combinatorial explosive problem computational tractable, a random sample consensus algorithm was employed for determining which sensors should be selected from the populations of 4 DoF poses associated with each sensor type, that was either randomly or uniformly deployed over regions in which the real sensors could be placed and provide useful perception data. With a reasonable sensor count, the proposed sensor placement system managed to compute a good sensor pose in a few seconds and suitable sensor constellations in a few minutes, which makes it suitable for active 3D perception operations and sensor layout optimization tasks.

Future work could include the testing of the proposed approach in conjunction with a object recognition system in order to reliably perform object tracking when an operator is manipulating a target object (by moving the sensor within the environment using a robotic arm).% Further research for fast object tracking recovery could include the modeling of the interaction between the target objects and a simulated hand, in order to keep an approximate object pose even if it is almost completely occluded.
