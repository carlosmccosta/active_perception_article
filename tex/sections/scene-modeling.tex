\section{3D scene modeling}\label{sec:modeling}

For being able to compute the surface coverage of a given set of simulated target objects that a given constellation of sensors can observe, it is necessary to model the 3D geometry of the scene objects. Moreover, the depth sensors must have a realistic data acquisition formulation that is representative of the real sensors. As such, the development of the proposed system started with the 3D modeling of the scene geometry, namely the environment objects and sensors 3D meshes using \gls{cad} systems. Later on, several types of depth sensors were modeled within the Gazebo simulator\footnote{\url{http://gazebosim.org}} in order to perform accurate 3D rendering of the scene and generate representative sensor data taking into consideration the specific characteristics of each type of sensor (such as image resolution, field of view and depth range) while also accounting for the occlusions that other objects in the environment might cause in relation to the target models that each sensor is trying to observe from its given view point. The next sections will present the modeling of the simulation worlds along with how the depth sensors were deployed within plausible regions of interest that take into account where the sensors can be placed in the real environment.


\subsection{Environment modeling}

For testing active perception and bin picking operations it was modeled 4 different simulation worlds. Within these environments it was deployed one or several target objects, which were instances of a starter motor \gls{cad} model with a unique green surface material that had no light effects, such as shading and shadows.

The first environment (shown in \cref{fig:active-perception-environment}), focused on an active perception task in which a starter motor was placed on top of a trolley and was being occluded by a human hand starting to grasp it. The goal of this environment was to simulate an active perception task, in which we may need to actively move a sensor within the environment to be able to keep tracking the pose of a given object (such is the case of objects that are being manipulated by humans in which the hands are creating significant occlusions and a static sensor constellation may not be able to observe enough surface geometry to be able to perform pose tracking with accuracy).

On the other 3 worlds, the main goal was similar but applied to bin picking operations. In this use case a static overhead camera can provide a rough estimation of the target objects and then based on the level of object recognition confidence and how significant are the occlusions, we may need to move a sensor attached to a robotic arm to several poses in order to gather further sensor data to increase the object recognition confidence and its pose estimation accuracy. In the first bin picking world, the starter motor was inside a large staking box together with an alternator and a differential gearbox (displayed in \cref{fig:bin-picking-environment}). The second bin picking environment is a variation of the first in which it was added 3 more differential gearboxes into the stacking box in order to significantly increase the occlusion of the target object (seen in \cref{fig:bin-picking-with-occlusions-environment}). Finally, the last bin picking environment (seen in \cref{fig:multiple-bin-picking-with-occlusions-environment}) is another variation of the first environment in which it was added 3 more target objects (one on top of the trolley and two on the middle shelves).

\begin{figure}[H]
	\centering
	\includegraphics[width=.24\textwidth]{environments/active-perception/gazebo-back}
%	\includegraphics[width=.24\textwidth]{environments/active-perception/gazebo-top}
	\includegraphics[width=.24\textwidth]{environments/active-perception/rviz-back-left}
%	\includegraphics[width=.24\textwidth]{environments/active-perception/rviz-back-right}
	\caption{Environment for active perception of a starter motor being grasped by a human hand. The left image is showing a rendering from the Gazebo simulator with the target object in green color while the right image is displaying with blue spheres in Rviz the associated reference point cloud.}
	\label{fig:active-perception-environment}
\end{figure}

\begin{figure}[H]
	\centering
%	\includegraphics[width=.24\textwidth]{environments/bin-picking/gazebo-front}
	\includegraphics[width=.24\textwidth]{environments/bin-picking/gazebo-top}
%	\includegraphics[width=.24\textwidth]{environments/bin-picking/rviz-front}
	\includegraphics[width=.24\textwidth]{environments/bin-picking/rviz-top}
	\caption{Environment for bin picking of one starter motor that is inside a large staking box together with an alternator and a differential gearbox.}
	\label{fig:bin-picking-environment}
\end{figure}

\begin{figure}[H]
	\centering
%	\includegraphics[width=.24\textwidth]{environments/bin-picking-with-occlusions/gazebo-front}
	\includegraphics[width=.24\textwidth]{environments/bin-picking-with-occlusions/gazebo-top}
%	\includegraphics[width=.24\textwidth]{environments/bin-picking-with-occlusions/rviz-front}
	\includegraphics[width=.24\textwidth]{environments/bin-picking-with-occlusions/rviz-top}
	\caption{Environment for bin picking of one starter motor with 3 differential gearboxes causing occlusions.}
	\label{fig:bin-picking-with-occlusions-environment}
\end{figure}

\begin{figure}[H]
	\centering
	\includegraphics[width=.24\textwidth]{environments/multiple-bin-picking-with-occlusions/gazebo-front}
	\includegraphics[width=.24\textwidth]{environments/multiple-bin-picking-with-occlusions/gazebo-top}\\
	\includegraphics[width=.24\textwidth]{environments/multiple-bin-picking-with-occlusions/rviz-front-corner}
	\includegraphics[width=.24\textwidth]{environments/multiple-bin-picking-with-occlusions/rviz-top}
	\caption{Environment for picking 4 starter motors with multiple occlusions.}
	\label{fig:multiple-bin-picking-with-occlusions-environment}
\end{figure}


\subsection{Sensors modeling}

Over the years it was developed a wide range of technologies for performing environment sensing. From the passive image sensors to the active systems that probe the environment using projected patterns, lasers or \gls{tof} devices. Given that the goal of the proposed system was to perform active perception or environment monitoring, it was modeled 8 different types of depth sensors (shown in \cref{fig:sensors}) which relied on 3 types of environment sensing technologies. One of them was the Kinect XBox One \gls{tof} device, 5 were structured light sensors (such as the Asus Xtion Pro Live, the Ensenso N35, the Intel RealSense SR300, the Kinect XBox 360 and the Orbbec Astra) and 2 were stereo vision systems (namely the MultiSense S7 and the ZED stereo camera). Each of these depth sensors can be modeled using the pinhole camera model, which allows to specify the main unique characteristics of each sensor, such as the depth image resolution (width and height in pixels), its \gls{fov} (horizontal and vertical in radians) and the range in which the sensor can retrieve valid measurements (minimum and maximum in meters). Moreover, since this camera model is implemented in most 3D rendering engines and optimized in todays powerful \glspl{gpu}, the sensor data generation can be performed very fast and efficiently using 3D rendering \glspl{api} such as the \gls{opengl}. This is the case of the Gazebo simulator, that uses the Ogre3D\footnote{\url{http://www.ogre3d.org}} rendering engine which in turn relies on \gls{opengl}. Besides camera modeling, Gazebo also allows to simulate the sensor acquisition rate (specified as the number of depth images generated per second), which is usually higher on structured light sensors and lower on stereo vision systems.

\begin{figure}
	\centering
	\includegraphics[width=.47\textwidth]{sensors/sensors-gray-with-links}
	\caption{CAD models of the 3D sensors with the display of the depth image coordinate frames using the ROS convention of x-y-z -> forward-left-up. Name of the 3D sensors from top left to bottom right: Kinect XBox 360, Asus Xtion Pro Live, Ensenso N35, Kinect XBox One, Orbbec Astra, MultiSense S7, Intel RealSense SR300, ZED stereo camera.}
	\label{fig:sensors}
\end{figure}


\subsection{Sensors deployment}\label{subsec:sensors-deployment}

Finding the optimal sensor constellation that maximizes the observed surface area of a given set of target objects is a challenging combinatorial explosive problem when considering the presence of occlusions within the environment. As such, for making the estimation of the sensor disposition computational feasible, the 3D continuous space was populated with a given set of sensors within regions of interest while looking at a specified observation point (with the sensor roll either 0º or random). This approach allows to reduce the sensor pose estimation from a 6 \gls{dof} to a 4 \gls{dof} problem (x, y, z position plus the sensor rotation along the observation axis). Moreover, the continuous solution space with an infinite amount of observation view points is reduced to a bounded number in which the sensors can either be deployed uniformly or randomly inside regions of interest. This allows to sample the solution space with a reasonable and representative amount of simulated sensor data for computing a good enough sensor disposition for the problem at hand.

The proposed system allows the deployment of several populations of sensors within a simulated world. Each population contains a given number of sensors of the same type that can be deployed uniformly / randomly within a box / cylinder or in a grid / linear disposition. This allows to deploy the sensors in the simulated environment spaces that represent valid positions for the real sensors. For example, limiting the possible sensor view points to the walls and ceiling of a room, avoiding deploying sensors in areas in which they could not provide any valuable sensor data or in which they could not be physically placed due to spatial restrictions or safety reasons.

The populations of sensors that were deployed on the 4 simulation worlds were fine tuned to the particular goals of each test. For the active perception environment, 450 sensors were deployed close to the target object, on the top, right and back side of the trolley (shown in \cref{fig:sensors-deployment-active-perception-environment}). This was done to simulate the closest range in which a dynamically moving sensor attached to a robotic arm could move (taking into consideration the human safety and the sensor minimum measurement distance, that was 0.2 meters).

\begin{figure}[H]
	\centering
	\includegraphics[height=.22\textwidth]{sensor-deployment/active-perception/gazebo-front}\hspace{1em}
	\includegraphics[height=.22\textwidth]{sensor-deployment/active-perception/gazebo-top}
	\caption{Sensors deployment for the active perception environment.}
	\label{fig:sensors-deployment-active-perception-environment}
\end{figure}

For the single object bin picking environments, given that the target object was inside the stacking box, the sensors were deployed close to the target object (displayed in \cref{fig:sensors-deployment-bin-picking-environments}), but only on top of the trolley, on 3 layers (each with a different type of sensor). In the world with minimal occlusions it were deployed 100 sensors while in the world with significant occlusions it were deployed 300 sensors. The sensor density was increased because when there is a high amount of occlusions, the best views have tighter observation regions, which could be missed with a sparse sensor deployment.

For the multiple object bin picking environment, given that there were several target objects (1 inside the stacking box, 1 on top of the trolley and 2 on the shelves of the trolley), it were deployed 450 sensors across 7 populations (visualized in \cref{fig:sensors-deployment-multiple-bin-picking-environment}), 5 of them simulating fixed sensors on the walls and ceiling and 2 of them simulating dynamic sensors attached to a robotic arm above the trolley.

It should be noted that the visual sensor disposition shown in \cref{fig:sensors-deployment-active-perception-environment,fig:sensors-deployment-bin-picking-environments,fig:sensors-deployment-multiple-bin-picking-environment} was for human visual inspection only. During the sensor data generation, the 3D models of the sensors are hidden to avoid occlusion of the scene objects.

\begin{figure}[H]
	\centering
	\includegraphics[height=.28\textwidth]{sensor-deployment/bin-picking/gazebo-sensors}\hspace{1em}
	\includegraphics[height=.28\textwidth]{sensor-deployment/bin-picking-with-occlusions/gazebo-sensors}
	\caption{2 sensors deployments for the 1 object bin picking environments.}
	\label{fig:sensors-deployment-bin-picking-environments}
\end{figure}

\begin{figure}[H]
	\centering
	\includegraphics[height=.17\textwidth]{sensor-deployment/multiple-bin-picking-with-occlusions/gazebo-front}\hspace{0.25em}
	\includegraphics[height=.17\textwidth]{sensor-deployment/multiple-bin-picking-with-occlusions/gazebo-top}
	\caption{Sensors deployment for the multiple object bin picking environment.}
	\label{fig:sensors-deployment-multiple-bin-picking-environment}
\end{figure}
